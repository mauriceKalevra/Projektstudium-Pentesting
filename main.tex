% class definitions
\documentclass[12pt]{article}
\usepackage[ngerman,english]{babel}
\setlength{\parindent}{0em} 

% Packages

\usepackage[utf8]{inputenc}
\usepackage[ngerman]{babel}
\usepackage[T1]{fontenc}
\usepackage{graphicx}
\usepackage{lmodern}
\usepackage{tabto}
\usepackage{listings}
\usepackage{quoting} %
\usepackage{lipsum}
\usepackage[left, pagewise, edtable]{lineno}
\quotingsetup{font={itshape}, leftmargin=2em, rightmargin=0in, vskip=1ex}
\usepackage{framed} 
\usepackage{xcolor}
\usepackage{tcolorbox}
\usepackage{xcolor} 
\colorlet{shadecolor}{gray!25}
\definecolor{mshadecolor}{rgb}{0.7421875,0.7421875,0.7421875}

%myshaded
\newenvironment{myshaded}{\colorlet{shadecolor}{lightgray}\color{black}\begin{shaded}\begin{internallinenumbers}}{\end{internallinenumbers}\end{shaded}}

%bibtext
\usepackage[backend=biber, style=authoryear]{biblatex}
\addbibresource{quellen.bib}


% Front page

\title{ \small{Projektstudium Penetrationstest}
\vspace{3mm}
\\\textbf{\Large{Penetrationstest Report}}
\\\vspace{3mm}\small{SySS GmbH}}
\author{\small{Moritz Rupp, Dean Basic, Lukas Heinzelmann, Marius Hald}}
\date{Sommersemester 2022}



%Document start 
\begin{document}



\maketitle
\newpage
\tableofcontents
\newpage
\section{Executive Summary}
Dieses Dokument beschreibt die Ergebnisse des Penetrationstests für die SySS GmbH. Zweck des Penetrationstest war es ein Überblick über Sicherheit und Konfiguration des Gerätes 'Home-matic' zu bekommen. Dieses dient als Zentrale für Smart-home Geräte und steuert somit alle verbundenen Einheiten wie Heizkörper, Lichter, Türen etc. Für die Verwaltung wird eine Weboberfläche bereitgestellt.
Der Penetrationstest simuliert einen böswilligen Angriff mit dem Ziel:
\begin{itemize}
 \item Festzustellen ob ein Angreiffer sensitive Daten der Zentrale bzw. des Betreibers abgreifen kann
 \item Die Auswirkungen einer Sicherheitslücke auf Verfügbarkeit, Vertraulichkeit und Integrität der Anwendung.
\end{itemize}
Die Angriffe wurden unter verschiedenen Szenarien wie dem Angriff von außen als auch durch interne Kompromitierungsversuche durchgeführt.
\section{Summary of Results}
Im Verlauf des Penetrationstestes konnten mehrere Sicherheitslücken festgestellt werden. Darunter befinden sich 2 akkute Bedrohungen. Des weiteren wurden 4 Befunde mit mittleren Berdohungsstufen ausfindig gemacht.


\newpage
\section{Scope and Methods}
Für den Penetrationstest stellte uns der Auftraggeber, die SySS Gmbh kompletten zugang zu der Home matic Zentrale. Somit war es möglich einen internet White-box Test durchzuführen. Um weitere Bedrohungen ausfindig zu machen wurde zudem ein Black-Box test simuliert und durchgeführt. \\

\end{document}

